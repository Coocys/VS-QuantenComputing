\section{Einleitung2}
\label{sec:einleitung2}

\subsection{Drei Prinzipien des Quanten Computing}
\label{subsec:drei-prinzipien-des-quanten-computing}

Zusammenfassend lässt sich Quanten Computing auf 3 wesentliche Prinzipien herunterbrechen.\\\\
\textbullet Prinzip 1 - \textbf{Das Quantenregister}: Ein Quantenregister, das aus n-Qubits besteht, wird durch einen $2^n$-dimensionalen Vektorraum über komplexen Zahlen beschrieben.
Der Zustand eines solchen Registers ist eine Überlagerung (Superposition) aller möglichen Basiszustände.
Das bedeutet, dass das Register eine Kombination vieler möglicher Werte gleichzeitig annehmen kann.

Diese Fähigkeit der Superposition ist eine der Hauptstärken von Quantencomputern, da sie es ermöglichen, mehrere Berechnungen parallel durchzuführen.\\

\textbullet Prinzip 2 - \textbf{Rechenschritte}: Rechenschritte in einem Quantencomputer basieren auf unitären Transformationen.
Diese Transformationen sind umkehrbar, was bedeutet, dass die Berechnung ohne Informationsverlust rückgängig gemacht werden kann.
Jede Operation kann lokal beschrieben werden, wobei nur zwei Qubits gleichzeitig beteiligt sind.

Diese Reversibilität der Rechenschritte stellt einen fundamentalen Unterschied zu klassischen Computern dar, bei denen Informationen während der Berechnung verloren gehen können.\\

\textbullet Prinzip 3 - \textbf{Messungen}: Misst man den Zustand eines Quantenregisters, so erhält man als Ergebnis einen der Basiszustände mit einer Wahrscheinlichkeit, die durch die Amplitude dieses Zustands in der Superposition gegeben ist.
Die Messung verändert den Zustand des Systems auf den gemessenen Wert, sodass die ursprüngliche Superposition zerstört wird.\\

In diesen Prinzipien unterscheidet sich das Quanten Computing wesentlich von klassischen Computern.


\subsection{Verschränkung}
\label{subsec:verschraenkung}

Eine der interessantesten Eigenschaften von Quantenregistern ist die Verschränkung.
Bei der Verschränkung teilen sich zwei Qubits denselben Zustand.
Das heißt, messen wir den Zustand von Qubit 1, wissen wir auch sofort den Zustand von Qubit 2, ohne dieses gemessen zu haben.
Und was das Ganze noch faszinierender macht, selbst über große Entfernungen zwischen den verschränkten Qubits, bleibt die Eigenschaft der Verschränkung erhalten.
Dies bildet auch die Grundlage für die Quanten-Teleportation, auf die wir später noch zurückkommen.\\

Wie erzeugen wir eine solche Verschränkung?
Dazu betrachten wir exemplarisch ein Zwei-Bit-Register $\ket{b_{1}b_{2}}$ im Zustand $\ket{00}$.
Wir wenden auf das erste Bit die Hadamard-Transformation an und anschließend auf beide Bits die Operation CNOT\@.
\begin{equation}
    CNOT : \ket{x,y} \rightarrow \ket{x,y\oplus x}
\end{equation}

Das ergibt:
\begin{align}
    \ket{00} &\xrightarrow{H\otimes I_{2}} \frac{1}{\sqrt{2}}(\ket{0}+\ket{1})\ket{0} = \frac{1}{\sqrt{2}}(\ket{00}+\ket{10}) \\
    &\xrightarrow{CNOT} \frac{1}{\sqrt{2}}(\ket{00}+\ket{11})
\end{align}

Wenn wir nun das erste Bit messen, kommt mit einer Wahrscheinlichkeit 50\% das Ergebnis $\ket{0}$ mit dem Folgezustand $\ket{00}$ und mit einer Wahrscheinlichkeit von 50\% das Ergebnis $\ket{1}$ mit dem Folgezustand $\ket{11}$ heraus.
Wir wissen also nach der ersten Messung schon, bevor wir das zweite Qubit überhaupt gemessen haben, wie der Endzustand des Quantenregisters ist.
Und wie bereits zuvor erwähnt, bleibt diese Eigenschaft bei räumlicher Trennung der Qubits erhalten.
Hierbei ist auch zu erwähnen, dass es egal ist, welches der Qubits zuerst gemessen wird oder ob diese überhaupt gleichzeitig gemessen werden.
\begin{figure}[H]
    \centering
    \includegraphics[width=0.8\textwidth]{img/Verschränkung Definition}
    \caption{Definition Verschränkung}
    \label{fig:verschraenkung}
\end{figure}

Diesen Zustand nennt man auch Bell-Zustand.
Es gibt insgesamt 4 solcher Bell-Zustände, diese beschreiben verschränkte Bits mit einer starken Kopplung (maximal verschränkt).\\
Daraus resultierend gibt es auch Verschränkungen mit einer weniger starken Kopplung.
Ein solcher Zustand könnte beispielsweise so aussehen:
\begin{equation}
    \ket{\phi} = 0.9\ket{00} + 0.1\ket{11}
\end{equation}

Hier sind die Qubits auch wieder miteinander verschränkt, allerdings sind die Wahrscheinlichkeiten für die Messergebnisse ungleich verteilt.
Das heißt, wir bekommen mit einer Amplitude von 90\%, also sehr sicher, den Zustand $\ket{00}$ und nur mit 10\% den Zustand $\ket{11}$, also unsicher.
Diese weniger stark gekoppelten Qubits kommen in der Praxis häufiger vor, etwa durch äußere Einflüsse wie Rauschen oder Dekohärenz auf ehemals maximal verschränkte Qubits.
Das geht mit Leistungseinbußen einher, weshalb versucht wird, den Zustand der maximalen Verschränkung möglichst lange zu erhalten.


\subsection{Quantengatter \& Quantenschaltkreise}
\label{subsec:quantengatter-quantenschaltkreise}

„Klassische Schaltkreise bestehen aus Leitungen und Gattern.
Ganz analog bestehen Quantenschaltkreise aus Quantenleitungen und Quantengattern.
Jede Quantenleitung entspricht einem Quantenbit und ein Quantengatter führt eine unitäre Transformation aus“
\begin{figure}[H]
    \centering
    \includegraphics[width=0.35\textwidth]{img/Quantengatter Basic}
    \caption{Quantenschaltkreis}
    \label{fig:quantenschaltkreis}
\end{figure}

Wenn wir eine Berechnung mit mehreren Qubits ausführen, ergibt sich der Endzustand $\ket{x´, y´, z´}$ aus dem Tensorprodukt der einzelnen Gatter:
\begin{equation}
    (I_{2} \otimes W \otimes I_{2})(U \otimes V)\ket{x, y, z}
\end{equation}

\begin{figure}[H]
    \centering
    \includegraphics[width=0.35\textwidth]{img/Quantengatter 3er}
    \caption{Quantenschaltkreis mit Tensorprodukt}
    \label{fig:quantenschaltkreis-tensorprodukt}
\end{figure}

Da bei Quantenschaltkreisen die Umkehrbarkeit der Berechnungen garantiert werden muss, ist die Summe der Eingabe-Qubits = Summe der Ausgabe-Qubits und pro Gatter dürfen höchsten 3 Qubits einbezogen werden.
Um dies zu gewährleisten, nutzen wir das Toffoli-Gatter, welches im nächsten Kapitel erläutert wird.
Außerdem können Qubits nicht kopiert werden, deshalb dürfen sich die Quantenleitungen nicht verzweigen und das Ergebnis eines Gatters nicht mehrfach verwendet werden.
Auf den Grund dafür kommen wir später nochmal zurück.\\

Eine der wichtigsten Operationen in der Quanteninformatik ist die Negation, genauer die kontrollierte Negation CNOT\@.
Dieses Gatter wird für alle Quantenoperationen benötigt, so zum Beispiel bei der Verschränkung.
Darstellen lässt es sich wie folgt:
\begin{equation}
    CNOT : \ket{x,y} \rightarrow \ket{x,y\oplus x}
\end{equation}

oder als Matrix:
\begin{equation}
    CNOT =
    \begin{pmatrix}
        1 & 0 & 0 & 0 \\
        0 & 1 & 0 & 0 \\
        0 & 0 & 0 & 1 \\
        0 & 0 & 1 & 0
    \end{pmatrix}
\end{equation}

Dieses CNOT-Gatter negiert nur dann das zweite Qubit, wenn das erste Qubit im Zustand $\ket{1}$ ist.\\

Weitere wichtige Operationen sind die Hadamard-Transformation, die wir bereits kennen, sowie die Pauli-Matrizen.
Die Pauli-Matrizen negieren ebenfalls, mithilfe einer unitären Transformation auf einem Bit.
Die bekannteste ist der „Bitflip“ X:
\begin{equation}
    X =
    \begin{pmatrix}
        0 & 1 \\
        1 & 0
    \end{pmatrix}
\end{equation}

Die zwei weiteren Pauli-Matrizen sind der „Phasenflip“ Z:
\begin{equation}
    Z =
    \begin{pmatrix}
        1 & 0 \\
        0 & -1
    \end{pmatrix}
\end{equation}

und der „Y-Flip“ Y:
\begin{equation}
    Y =
    \begin{pmatrix}
        0 & -i \\
        i & 0
    \end{pmatrix}
\end{equation}

\begin{figure}[H]
    \centering
    \includegraphics[width=0.5\textwidth]{img/Quantengatter Pauli_Matrizen}
    \caption{Pauli-Matrizen}
    \label{fig:pauli-matrizen}
\end{figure}

Durch die Kombination dieser drei Gatter CNOT, Hadamard-Transformation, sowie Pauli-Matrizen lassen sich alle Quantenberechnungen abbilden.
Sie bilden die grundlegendsten Rechenoperationen eines Quantencomputers.
Mit einem wesentlichen Unterschied zu logischen Gattern: bei diesen lässt sich der Endzustand nicht unbedingt wieder in den Anfangszustand überführen.
Bei Quantengattern ist dies eine zwingende Voraussetzung, sie müssen umkehrbar sein.


\subsection{Umkehrbare Berechnungen}
\label{subsec:umkehrbare-berechnungen}

Wie bereits zuvor erwähnt, muss jede Rechenoperation eines Quantencomputers umkehrbar sein.
Es dürfen also keine Informationen gelöscht werden, wie es beispielsweise bei der Anwendung einer logischen AND-Operation passiert: aus zwei Eingabewerten wird ein Ausgabewert kombiniert.
So folgt aus 1 AND 0 = 0, jedoch ebenfalls aus 0 AND 0 = 0.
Sehen wir den Endzustand 0, wissen wir also nicht, welchen Zustand die beiden Bits zu Beginn hatten.
Das bedeutet, wir können Quantenrechenprozesse nicht auf dieselbe Art verarbeiten wie klassische Rechenprozesse.\\

Allerding kann jede klassische Operation in eine umkehrbare Operation umgewandelt werden.
Veranschaulichen wir uns dies anhand des Toffoli-Gatters.\\

Das Toffoli-Gatter ist ein universelles, umkehrbares Gatter, welches AND, OR und NOT Operationen ersetzen kann.
Es besteht aus drei Eingabebits a, b, c und drei Ausgabebits a, b und $c\oplus(a\land b)$:
\begin{figure}[H]
    \centering
    \includegraphics[width=0.35\textwidth]{img/Umkehrbare_Berechnungen Toffoli}
    \caption{Toffoli-Gatter}
    \label{fig:toffoli-gatter}
\end{figure}

Exemplarisch für AND: Wollen wir zum Beispiel ein Bit negieren, mit der Eingabe (a, b, c) = (1, 1, 0).
Dann ergibt die Rechnung über das Toffoli-Gatter (1, 1, 1), denn $(1\land 1) \oplus 0 = 1$.
Hier ist nun auch zusehen, dass die Berechnung umkehrbar ist, wenn wir die Schritte einmal rückwärts gehen.
Dies ist genauso für die anderen Operatoren möglich, bedarf gegebenenfalls nur Umformung und den Einsatz mehrere Toffoli-Gatter.\\

Auf diese Weise können wir jede klassische Rechenoperation umkehrbar gestalten, sodass diese auch für unsere Quantenschaltkreise nutzbar sind.


\subsection{Gestörte Berechnungen}
\label{subsec:gestoerte-berechnungen}

Wie wir bisher sehen konnte, scheint alles, was wir mit klassischen Computern berechnen, ebenfalls genauso effizient mit Quantencomputern berechnen zu lassen.
Warum also steht nicht bei jedem ein Quantenrechner zu Hause (wenn wir die Kosten mal außen vor lassen)?\\

Der Grund dafür liegt in der Fehleranfälligkeit.
Während klassische Bits die Zustände 0 oder 1 abbilden, können Quantenbits bis zur Messung $\ket{0}$ oder $\ket{1}$ oder jeden Zustand dazwischen annehmen.
Verdeutlichen wir, was das in der Praxis bedeutet: Nehmen wir an, der Zustand 0 eines klassischen Bits wird durch 0V und der Zustand 1 durch 5V dargestellt.
Nun kann es zu Spannungsschwankungen kommen und wir geben 4V statt 5V\@.
Da wir aber keinen Zustand für 4V haben, aber nur 2 Zustände abbilden können, können wir auch sagen, der Zustand 0 wir durch eine Spannung <2,5V abgebildet und der Zustand 1 durch >2,5V.
So liefert der Rechner uns auch weiterhin ein zuverlässiges Ergebnis, trotz Störung.\\

Die ist allerdings nicht für Quantenbits möglich, da wir nicht nur diese zwei Zustände abbilden.
Was bedeutet also hier eine Spannungsschwankung von 5V auf 4V?
Das wissen wir nicht, da bis zur Messung der Zustand nicht feststeht, so können wir also auch kein zuverlässiges Ergebnis mehr liefern.
So können selbst kleinste Störungen Quantenberechnungen verfälschen.
Was jedoch zu beachten ist, bei einem gestörten Quantengitter addiert sich der Fehler.
Das heißt, der Fehler wird in nachfolgenden Berechnungen nicht weiter verstärkt.\\

Um die Frage vom Beginn des Kapitels also nochmal aufzunehmen: Der Grund weshalb nicht jeder einen Quantencomputer zu Hause stehen hat, ist die Fehleranfälligkeit der Berechnungen.
Es bedarf großen Aufwands solche Fehler zu vermeiden und zu korrigieren.


\subsection{Grenzen}
\label{subsec:grenzen}
\textcolor{red}{\textbf{INCOMPLETE}}\\

Angenommen in 10 Jahren ist das Problem der Dekohärenz gelöst, werden Quanten Computer die Lösung aller mathematischen Probleme sein?
Obwohl Quanten Computing noch mitten in der Entwicklung stecken, lässt sich jetzt schon absehen: Die Antwort darauf ist „Nein“.\\

Eine der zentralen Fragen der Komplexitätstheorie beschäftigt sich mit NP-vollständigen Problemen.
Sie beschreibt Probleme, welche sich zwar leicht überprüfen lassen, doch algorithmisch mindestens eine exponentielle (O($2^{n}$), häufig sogar fakultative (O(n!)) Laufzeit haben.
Und würde man für ein NP-vollständiges Problem eine Lösung finden, also ein Algorithmus mit polynomieller O($n^{x}$) Laufzeit, so könnten alle dieser Probleme darauf umgeformt und gelöst werden.\\

Doch selbst Quanten Computer scheinen keine Lösung für diese Probleme finden zu können.\\





















































