\section{Quantenteleportation}\label{sec:quantum-teleportation}

\subsection{Einführung}\label{subsec:introduction}
Quantenteleportation ist ein bahnbrechendes Phänomen, das die Übertragung von Quanteninformationen
zwischen zwei entfernten Orten ermöglicht, ohne dass das Teilchen oder Objekt, das die Information trägt, physisch bewegt wird.
Im Gegensatz zur theoretischen klassischen Teleportation, bei der es um den Transport von Materie oder Energie geht,
konzentriert sich die Quantenteleportation auf die Übertragung von Quantenzuständen.
Dieser Prozess nutzt die Prinzipien der Quantenmechanik, einschließlich Verschränkung, Superposition,
und das No-Cloning-Theorem, das wir im vorherigen Abschnitt erläutert haben,
um den Zustand eines Quantenobjekts (z.B.\ eines Photons oder eines Elektrons) von einem Ort zum anderen zu übertragen,
ohne Rücksicht auf die Entfernung.\\

Der Begriff ``Quantenteleportation'' kann etwas irreführend sein, da bei diesem Prozess keine eigentliche Materie teleportiert wird.
Was stattdessen ``teleportiert'' wird, ist die Information über den Quantenzustand eines Teilchens.
Der Schlüssel zur Quantenteleportation ist die Quantenverschränkung,
ein Phänomen, bei dem zwei oder mehr Teilchen so miteinander korrelieren, dass der Zustand des einen Teilchens den Zustand des anderen augenblicklich
den Zustand des anderen Teilchens beeinflusst, unabhängig davon, wie weit sie voneinander entfernt sind.
Diese ``gespenstische Fernwirkung'', wie Albert Einstein sie nannte,
ermöglicht die Übertragung von Quanteninformation zwischen weit entfernten Parteien, ohne dass die oft fragilen Quantenzustände selbst transportiert werden müssen.

\subsection{Aufbau}\label{subsec:setup}

Es muss ein Quanten-Verschränkungspaar vorhanden sein.
Dieses Paar muss sich im Bell-Zustand\cite{quantuminfocambridge} befinden, um sicherzustellen, dass die Messung des einen den Zustand des anderen beeinflusst.
Dieses Paar wird in der Regel durch einen Prozess wie
Spontane parametrische Abwärtsumwandlung\cite{couteau2018spontaneous} erzeugt und derzeit
werden in der Regel einfache Photonen oder Ionen verwendet.\\


Außerdem müssen sich an den Endpunkten der Teleportation zwei Parteien befinden, im Folgenden Alice und Bob genannt, die beide
die beide in der Lage sind, mit Quantensystemen zu interagieren und Messungen vorzunehmen, was normalerweise einen Quantencomputer mit begrenzter Funktionalität bedeutet.
Schließlich muss ein klassischer Kommunikationskanal zwischen Alice und Bob bestehen.
Der Schlüssel für die Quantenteleportation ist hier, dass dieser Kanal nicht in der Lage sein muss, Quantenzustände zu transportieren - dafür ist die Teleportation gedacht -, sondern nur herkömmliche Bits.

%<Diagramm Aufbau>

\subsection{Vorgang}\label{subsec:process}

Nachdem der Aufbau abgeschlossen ist, stellt sich nun die Frage, was tatsächlich getan werden muss, um die Quantenteleportation durchzuführen.
Dies kann in mehrere Schritte aufgeteilt werden.

\subsubsection{Alices Bell Zustand Messung}
Der erste Schritt, den Alice durchführt, ist die Bell-State-Messung, die zwei wichtige Teilschritte umfasst.

Zunächst kombiniert Alice das Teilchen, das sie teleportieren möchte, mit ihrer Hälfte des verschränkten Paares.
Dies geschieht in der Regel mithilfe eines Strahlenteilers oder eines Interferometers, um die beiden Teilchen in einen Superpositionszustand zu versetzen.

Zweitens misst Alice die beiden kombinierten Teilchen in der Bell-Basis, die die Teilchen aus ihren vier möglichen Zuständen in einen einzigen zusammenfallen lässt.
Diese Messung ist entscheidend, denn sie bestimmt, wie Bob sein Teilchen anpassen muss, um den Zustand wiederherzustellen, den
Alice teleportiert.
\subsubsection{Klassische Kommunikation}
Als Nächstes verwendet Alice ihren klassischen Kommunikationskanal, um ihre Messung an Bob zu senden.
Dies erfordert die Übertragung von nur zwei Bits (den beobachteten Zustand des zu teleportierenden Qubits und den beobachteten Zustand des
des verschränkten Paares), was für keinen Kommunikationskanal ein Problem darstellen sollte.
Diese geringe Bandbreitenanforderung in der klassischen Kommunikation macht in der Tat mehrere Kommunikationskanäle verfügbar
die normalerweise wegen ihrer geringen Bandbreite nicht infrage kämen, insbesondere wenn die Teleportation über große
Entfernungen stattfinden soll.
Diese Information reicht jedoch aus, um Bob zu instruieren, wie er sein eigenes System einstellen muss.
\subsubsection{Bobs Quantenoperation}
Nun, da Bob die klassische Messung erhalten hat, muss er eine oder beide (je nach Messung) der folgenden Methoden anwenden: Das
Pauli-X-Gatter für einen Bit-Flip oder das Pauli-Z-Gatter für einen Phasen-Flip.
Durch diese Operation wird sein Quantenzustand in denselben Zustand versetzt, den Alices Quantenteilchen hatten, bevor ihre
Messung die Superposition kollabierte.
Hier ist auch noch einmal darauf hinzuweisen, dass Bob den Zustand erst erstellt, nachdem Alice ihn mit der Messung
bereits zerstört hat - wie das No-Cloning-Theorem besagt\ref{sec:no-cloning} wird der Zustand nie kopiert, sondern nur
übertragen.

Sobald Bob dies getan hat, ist die Quantenteleportation abgeschlossen.
\subsection{Mathematik}\label{subsec:proof2}

\subsubsection{Verschränkung und Bell-Zustände}

Zu Beginn des Prozesses haben wir zwei Teilchen (Teilchen 2 und 3), die sich an den Orten A und B befinden.
Diese Teilchen werden in einem verschränkten Zustand (Bell-Zustand) erzeugt.
Ein Bell-Zustand ist eine der vier möglichen maximal verschränkten Zustände, die ein Paar von Quantenobjekten haben kann.
Ein Beispiel für einen solchen Zustand ist:

\begin{equation}
|\Phi^+\rangle = \frac{1}{\sqrt{2}} \left( |00\rangle + |11\rangle \right)
\end{equation}

Dieser Zustand wird zwischen den beiden Teilchen 2 und 3 geteilt, wobei Teilchen 2 bei A und Teilchen 3 bei B ist.

\subsubsection{Der zu teleportierende Zustand}

Nun nehmen wir an, dass wir den Zustand eines dritten Teilchens (Teilchen 1) teleportieren möchten, das sich am Ort A befindet. Der Zustand des Teilchens 1 kann allgemein als:

\[
|\psi\rangle_1 = \alpha |0\rangle + \beta |1\rangle
\]

ausgedrückt werden, wobei \(\alpha\) und \(\beta\) komplexe Zahlen sind, die den Zustand beschreiben.

\subsubsection{Verschränkung und Messung}

Der gesamte Zustand des Systems (Teilchen 1, 2 und 3) kann als Produktzustand von Teilchen 1 und dem verschränkten Zustand von Teilchen 2 und 3 beschrieben werden:

\[
|\Psi\rangle_{123} = |\psi\rangle_1 \otimes |\Phi^+\rangle_{23} = \frac{1}{\sqrt{2}} \left( \alpha |0\rangle_1 + \beta |1\rangle_1 \right) \otimes \left( |00\rangle_{23} + |11\rangle_{23} \right)
\]

Durch Anwenden der Distributivität ergibt sich:

\begin{gather*}
    |\Psi\rangle_{123} = \frac{1}{\sqrt{2}} \left( \alpha |0\rangle_1 \otimes (|00\rangle_{23} + |11\rangle_{23}) + \beta |1\rangle_1 \otimes (|00\rangle_{23} + |11\rangle_{23}) \right)\\
    |\Psi\rangle_{123} = \frac{1}{\sqrt{2}} \left( \alpha |000\rangle_{123} + \alpha |011\rangle_{123} + \beta |100\rangle_{123} + \beta |111\rangle_{123} \right)\\
\end{gather*}

Nun wird eine Bell-Zustandsmessung auf den Teilchen 1 und 2 durchgeführt, die den Zustand des Systems in einen der vier Bell-Zustände projiziert. Die Messung ist zufällig, und die Ergebnisse können durch die folgenden Zustände beschrieben werden:

\[
|\Phi^+\rangle_{12}, |\Phi^-\rangle_{12}, |\Psi^+\rangle_{12}, |\Psi^-\rangle_{12}
\]

\subsubsection*{Klassische Kommunikation und Zustandserstellung}

Nachdem die Messung durchgeführt wurde, sendet der Ort A das Messresultat an Ort B über einen klassischen Kanal. Anhand der Nachricht kann Ort B den Zustand des Teilchens 3 (das ursprünglich am Ort B war) in den gewünschten Zustand \(|\psi\rangle_1\) transformieren. Dazu wird eine der folgenden Operationen durchgeführt, abhängig von der Messung, die an Ort A durchgeführt wurde:

\begin{gather*}
    |0\rangle_3 \quad \text{(falls Messung das Ergebnis } |\Phi^+\rangle\text{ ergibt)}\\
    X|0\rangle_3 \quad \text{(falls Messung das Ergebnis } |\Phi^-\rangle\text{ ergibt)}\\
    Z|0\rangle_3 \quad \text{(falls Messung das Ergebnis } |\Psi^+\rangle\text{ ergibt)}\\
    XZ|0\rangle_3 \quad \text{(falls Messung das Ergebnis } |\Psi^-\rangle\text{ ergibt)}\\
\end{gather*}

Durch diese Operationen wird der Zustand des Teilchens 3 in den ursprünglichen Zustand von Teilchen 1 (\( \alpha |0\rangle + \beta |1\rangle \)) überführt.


\subsection{Herausforderungen}\label{subsec:challenges}
Obwohl die Quantenteleportation ein vielversprechender Forschungszweig ist, gibt es noch einige Herausforderungen zu bewältigen

\subsubsection{Verschränkung Erzeugen und Erhalten}
Eine der größten Herausforderung ist die Erzeugung und Erhaltung eines verschränkten Systems zwischen dem Start- und Zielpunkt
der Teleportation.
Die Teleportation ist zwar theoretisch nicht durch Entfernung begrenzt, aber je größer die Entfernung zwischen den Orten
des do schwieriger ist es die verschränkten Teilchen aufzuteilen, ohne die Verschränkung zu beschädigen.\\

Ein Lösungsansatz sind hier Quantenrepeater: Spezialisierte Geräte die die Entfernung zwischen direkt verschränkten Teilchen reduzieren,
indem sie diese nur zwischen Repeater Stationen aufteilen müssen.
In der Station werden dann mithilfe von Entanglement Swapping zwei Verbindungen des Repeaters verschränkt.

\subsubsection{Klassische Kommunikation}
Auch wenn die benötigte Bandbreite der klassischen Kommunikation minimal ist, muss trotzdem ein Kommunikationskanal
existieren.
Das hat zwei signifikante Nachteile: Zum einen ist die klassische Kommunikation auf die Lichtgeschwindigkeit begrenzt,
was ein Geschwindigkeitslimit für die Teleportation erzeugt, auch wenn die ``spukhafte Fernwirkung'' der Quantenmechanik
schneller passieren könnte\cite{hensen2015loophole}.
Zum anderen sind klassische Kommunikationskanäle anfällig für Observation - ein Angreifer kann zwar
ohne das verschränkte Teilchen den Quantenzustand nicht reproduzieren ist aber in der Lage festzustellen, dass die
Kommunikation stattgefunden hat.
Ebenfalls könnte der Angreifer auch die Kommunikation stören, was zwar bei geeigneten Protokollen den Teilnehmern
offensichtlich ist aber trotzdem eine Schwachstelle darstellt.
Die einzige bekannte Lösung ist hier ein robustes klassisches Kommunikationssystem, was für andere Kommunikationszwecke
bereits aufgebaut ist oder wird aber leider die Lichtgeschwindigkeitslimitation nicht umgehen kann.


\subsubsection{Skalierbarkeit}

Ein weiteres bedeutendes Problem der Quantenteleportation ist die Skalierbarkeit.
Während die Quantenteleportation in kleinen, kontrollierten Systemen von einigen wenigen Qubits relativ einfach durchgeführt werden kann,
stellt die Skalierung auf größere Netzwerke und damit nützliche Datenmengen und die Integration in reale Kommunikationssysteme eine enorme Herausforderung dar.
Um Quantenteleportation praktisch nutzen zu können würde es ein großflächiges System von Kommunikationskomponenten
benötigen.
Die Erstellung dieser bräuchte Quantentechnologien in einer Menge in der diese momentan einfach weder technisch möglich
noch finanziell tragbar, wenn man bestehende Preise hochrechnet.\\

Hier ist allerdings die Erwartung, dass weitere Forschung dieses Problem beheben wird.
Es wird sowohl an günstigeren Methoden zur Erstellung von verschränkten Systemen und der Stabilisierung dieser
als auch an der Erstellung von Quantenkomponenten rege geforscht.
Auch an Forschungsbudget fehlt es hier nicht, da alle große Technologiefirmen an Quantentechnologie forschen.

\subsubsection{Fehlerquellen und Messgenauigkeit}
Die Messgenauigkeit ist entscheidend für die erfolgreiche Durchführung der Quantenteleportation.
Eine fehlerhafte Messung der Bell-Zustände kann dazu führen, dass der teleportierte Zustand nicht korrekt wiederhergestellt werden kann.
Fehlerquellen können in den Messinstrumenten, in der Kommunikation oder auch in der Quantenverschränkung selbst liegen.

Auch hier wird rege geforscht da alle drei Aspekte nicht eigen zur Quantenteleportation sind, sondern nahezu alle
Quantenoperationen betreffen.



