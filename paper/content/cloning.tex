\section{No Cloning Theorem} \label{sec:no-cloning}

Einer der faszinierendsten Aspekte der Quantenmechanik ist, dass es unmöglich ist
einen beliebigen unbekannten Quantenzustand perfekt zu duplizieren.
Dieses Konzept ist mit dem No-Cloning-Theorem beschrieben, einem grundlegenden Ergebnis der Quanteninformationstheorie.
Das Theorem hat nicht nur tiefgreifende Auswirkungen auf die Quanteninformatik und die Quantenkommunikation,
sondern auch für unser Verständnis der Natur der Information in Quantensystemen.\\

In der klassischen Physik ist die Vervielfältigung von Informationen einfach:
Ein Kopiergerät kann ein Dokument vervielfältigen, ohne das Original zu verändern.
In der Quantenmechanik wird dieser einfache Vorgang jedoch zu einer nicht trivialen und verbotenen Operation.
Das No-Cloning-Theorem besagt, dass es keine universelle Quantenoperation gibt
die eine identische Kopie eines beliebigen unbekannten Quantenzustands erzeugen kann.\\

In diesem Abschnitt werden wir das No-Cloning-Theorem aus verschiedenen Blickwinkeln betrachten, seinen Beweis diskutieren,
seine Konsequenzen untersuchen und verstehen, wie es die Landschaft der Quantentechnologien prägt.

\subsection{Definition}\label{subsec:definition}
Das No-Cloning-Theorem besagt, dass es unmöglich ist, eine identische Kopie eines unbekannten Quantenzustands zu erzeugen.
Mathematisch gesehen gibt es keinen unitären Operator $U$, der die folgende Bedingung erfüllt:
\begin{equation}
    U(\ket{\psi}\otimes\ket{e})=\ket{\psi}\otimes\ket{\psi}\label {eq:cloning_assertion}
\end{equation}
wobei $\ket{\psi}$ ein beliebiger Quantenzustand und $\ket{e}$ in Hilfszustand ist, der überschrieben werden soll.
Diese Gleichung drückt die Idee aus, dass, wenn wir den Operator $U$ auf den Zustand $\ket{\psi}$ anwenden
(kombiniert mit einem Hilfszustand), zwei Kopien des ursprünglichen Zustands entstehen sollten.
Das Theorem sagt uns, dass es für beliebige $\ket{\psi}$ keinen solchen Operator geben kann,
weil sich die Quanteninformation grundlegend von der klassischen Information unterscheidet.
\subsection{Beweis}\label{subsec:proof}
Als grundlegende Schlussfolgerung der Quantenmechanik gibt es mehrere Beweise für dieses Theorem.
Ein solcher Beweis, ein Widerspruchsbeweis, der der Einfachheit halber gewählt wurde, kann wie folgt definiert werden.

Zuerst nehmen wir zwei beliebige Quantenzustände, $\ket{\psi_1}$ und $\ket{\psi_2}$.
Dann werden diese beiden Zustände mit dem Klon operator $U$ geklont, den wir im vorigen Abschnitt definiert haben
und von dem wir für unseren Widerspruchsbeweis nun annehmen, dass er existiert.
\begin{equation}
\begin{split}
    U(\ket{\psi_1} \otimes \ket{e}) = \ket{\psi_1} \otimes \ket{\psi_1}\\
    U(\ket{\psi_2} \otimes \ket{e}) = \ket{\psi_2} \otimes \ket{\psi_2}
\end{split}\label{eq:clone_two_states}
\end{equation}

Mit dieser Information haben wir jetzt zwei Möglichkeiten, das Skalarprodukt von
$\braket{U(\psi_1 \otimes e)}{U(\psi_2 \otimes e)}$ zu schreiben.
Die eine verwendet das Ergebnis der vorherigen Gleichung, während die andere die Tatsache nutzt,
dass Quantenoperationen das Skalarprodukt ihrer Eingaben bewahren\footnote{Vgl: Postulates for general quantum mechanics, Segal, Irving E, 1947, S. 930--948.}.

\begin{equation}
\begin{split}
    \braket{U(\psi_1 \otimes e)}{U(\psi_2 \otimes e)} = \braket{\psi_1 \otimes \psi_1}{\psi_2 \otimes \psi_2}\\
    \braket{U(\psi_1 \otimes e)}{U(\psi_2 \otimes e)} = \braket{\psi_1 \otimes e}{\psi_2 \otimes e}
\end{split}\label{eq:two_scalar_products}
\end{equation}

Einfache Ersetzung führt dann zur folgenden Gleichung
\begin{equation}
    \braket{\psi_1 \otimes \psi_1}{\psi_2 \otimes \psi_2} =
    \braket{\psi_1 \otimes e}{\psi_2 \otimes e}
\label{eq:tensor_equals}
\end{equation}

Da Tensor und Skalarprodukte kompatible sind\footnote{Siehe Fußnote 8} simplifiziert das weiter zu:

\begin{equation}
    \braket{\psi_1}{\psi_2}\braket{\psi_1}{\psi_2}=\braket{\psi_1}{\psi_2}\braket{e}{e}
    \label{eq:scalar_equals}
\end{equation}

Und schließlich, weil für jeden Zustand $\ket{e}$ die Gleichung $\braket{e}{e} = 1$ gilt, kommen wir zu dieser Gleichung:

\begin{equation}
    \braket{\psi_1}{\psi_2}^2 = \braket{\psi_1}{\psi_2}
    \label{eq:square_equals_single}
\end{equation}

Es sollte nun offensichtlich sein, dass diese Gleichung nur zwei Lösungen hat:
$\braket{\psi_1}{\psi_2} = 1$ or $\braket{\psi_1}{\psi_2} = 0$.
Die erste Lösung impliziert, dass $\psi_1 = \psi_2$, was für eine allgemeine Klon-Operation nicht hilfreich ist, erlaubt aber eine
Operation, die Kopien eines bestimmten Zustands erzeugt.
Das ist vor allem nützlich um Quantensysteme in einen bekannten Zustand zu initialisieren.
Die zweite Lösung erlaubt zwar, dass sich $\psi_1$ und $\psi_2$ unterscheiden, und ist damit auf den ersten Blick
vielversprechend, verlangt aber immer noch, dass die beiden Zustände orthogonal zueinander sind.
Das ist zwar weniger restriktiv als die erste Lösung erlaubt aber immer noch nur eine Operation die eine bestimmte Klasse
von Zuständen kopieren kann von der alle Mitglieder zueinander orthogonal sind.
Somit ist auch mit dieser Lösung keine allgemeine Klon-Operation möglich.
\subsection{Folgen}\label{subsec:implications}
Das No-Cloning-Theorem hat weitreichende Auswirkungen auf eine Reihe von Gebieten der
Quantenmechanik und Quanteninformatik.

\subsubsection{Quantenkommunikation}
In der klassischen Informationstheorie ist das Kopieren von Daten eine zentrale Technik, um Information zu vervielfältigen und zu übertragen.
Bei klassischen Bits kann man exakt den gleichen Wert duplizieren, indem man eine Kopie eines Bits erstellt.
Im Gegensatz dazu verhindert das No-Cloning-Theorem das Erstellen von exakten Kopien von Quantenbits (Qubits).\\

Das bedeutet, dass im Bereich der Quantenkommunikation, insbesondere bei der Quantenkryptographie, ein Angreifer,
der versucht die Quanteninformation abzufangen oder zu kopieren, in der Regel Fehler einführen wird, die entdeckt werden können.
Dieses Prinzip bildet die Grundlage für Sicherheitsprotokolle wie Quantum Key Distribution (QKD),
bei denen ein Abhörversuch die Übertragung zerstören würde und somit leicht zu erkennen ist.

\subsubsection{Schutz der Quanteninformation}
Da das No-Cloning-Theorem das exakte Kopieren eines unbekannten Zustands verbietet,
wird auch die Quanteninformation von Natur aus gegen bestimmte Arten von Angriffen geschützt.
In klassischen Computersystemen kann ein Angreifer beliebig viele Kopien von Information erstellen,
um sie zu analysieren und gegebenenfalls zu entschlüsseln.
In einem quantenmechanischen System jedoch kann ein Datenexfiltrationsversuch oder das Kopieren eines Zustands
nicht ohne weiteres erfolgen, ohne dass der Versuch des Kopierens den Zustand verändert
und die Quanteninformation damit entwertet wird.\\

Ein gutes Beispiel für diese Art von Sicherheit ist das BB84-Protokoll für Quantenkryptographie.
Bei der Quantenverschlüsselung wird eine Nachricht durch verschränkte Quantenbits übertragen.
Jeder Versuch, die Nachricht zu kopieren oder abzufangen, verändert den Zustand der Qubits und wird vom Empfänger erkannt.

\subsubsection{Design von Algorithmen}

Das No-Cloning-Theorem hat auch tiefgehende Auswirkungen auf das Quantencomputing.
In klassischen Computern ist das Kopieren von Informationen eine grundlegende Technik, die in vielen Algorithmen und Protokollen verwendet wird.
Quantencomputer hingegen können keine exakten Kopien eines Zustands herstellen, was bedeutet,
dass traditionelle Techniken wie fehlerkorrigierende Codes, die in klassischen Computern üblich sind, in der Quantenwelt nicht direkt anwendbar sind.\\

Allerdings existieren spezielle Quantenfehlerkorrekturcodes, die darauf ausgelegt sind, Fehler zu korrigieren,
die durch das Fehlen einer exakten Kopierbarkeit von Quanteninformation entstehen.
Diese Codes erfordern jedoch eine zusätzliche Anzahl von Qubits und eine komplexe Fehlerkorrekturstrategie,
was das Quantencomputing technisch anspruchsvoll macht.
Trotzdem sind Quantenfehlerkorrekturmethoden von entscheidender Bedeutung für die zukünftige Skalierbarkeit
und Zuverlässigkeit von Quantencomputern und werden später in diesem Artikel genauer beschrieben[\ref{sub:quantum_error_correction}].

\begin{tcolorbox}[title=Kommentar,
    title filled=false,
    colback=cyan!5!white,
    colframe=cyan!75!black]
Verschiedene Messmethoden
Jeder Quantenalgorithmus ist auch grundsätzlich anders als ein klassischer Algorithmus, da Operationen anders implementiert
werden müssen, auch wenn die Theorie des Algorithmus identisch zu einem klassischen Algorithmus ist.
    Dieser Aspekt wird hier allerdings nicht näher behandelt werden.
\end{tcolorbox}

\subsubsection{Speicherung von Quanteninformation}
Das No-Cloning-Theorem hat weitreichende Konsequenzen für die Quanteninformationstheorie,
insbesondere für die Konzepte der Informationsspeicherung und -übertragung.
Die Unmöglichkeit des Klonens ist eng mit den grundlegenden Prinzipien der Quantenmechanik wie Überlagerung und Verschränkung verknüpft.
Sie hindert die Schaffung von perfekten Kopien von Quanteninformation und erfordert, dass Information auf neue, kreative Weise verarbeitet und gespeichert wird.\\

Ein interessantes Beispiel sind die Quantenlogikgatter, die in Quantencomputern verwendet werden.
Diese Gatter müssen mit den Einschränkungen des No-Cloning-Theorems arbeiten und können keine klassischen,
deterministischen Kopien erzeugen, sondern müssen die Quanteninformation in verschränkten oder überlagerten Zuständen manipulieren.

\subsubsection{Messungen und Messgenauigkeit}

In der Quantenmetrologie, die sich mit der präzisen Messung von quantenmechanischen Systemen beschäftigt,
beeinflusst das No-Cloning-Theorem ebenfalls die Art und Weise, wie Messungen durchgeführt werden können.
Da das exakte Kopieren von Zuständen nicht möglich ist, kann das Präzisionsmaß für Messungen nicht durch das
Vervielfachen von Messinstrumenten oder durch das Erstellen von Kopien von Quantenobjekten verbessert werden.
Stattdessen wird die Quantenmessung durch andere Techniken wie
Quanteninterferometrie und den Einsatz von verschränkten Zuständen optimiert.



