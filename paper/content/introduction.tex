\section{Introduction} \label{sec:introduction}
The goal of this paper is to explore the opportunities presented by quantum computing and its difficulties,
both in terms of information theoretical Aspects inherent to the concept and in practical aspects of its current
implementations.

Without first introducing the differences between a quantum computer and classical computer most Readers would however
lack the required context to understand the following paper and as such, this introduction will explain the most
critical aspects.

\subsection{Differences between classical and quantum computing theory} \label{subsec:diffclassquanttheory}
The first critical difference between the two computing methods in their theory is their most simple concept of data:
The classical computing theory operates on bits, a simple scalar value that is always one of a finite set of values
(typically either 1 or 0), with their only connection the one imposed by the algorithms in use.
It is also deterministic, with no true randomness, merely internal state that cannot be deduced from outside the system
and can be used to generate unpredictable, but not random values.

In contrast, quantum computing theory operates on the qubit, a superposition of 1 and 0 that is either of the values
with a given probability.
These qubits are typically either represented as a complex number or a unit vector.
Furthermore, for must practical application the qubits in use are not independent but entangled with some number of
other qubits, forming an entangled register, where the superpositions of the individual qubits are connected.
The probabilistic nature of qubits also gives quantum computing access to true randomness without any need for the
complex pseudo random number generators of classical computing.
Indeed, most quantum computing focuses on eliminating the randomness so that the odds of all correct results combined
come as close to 1 as possible, while the likelihood of the other results are reduced to near zero.

Another difference appears due to the physical requirements of superposition: All operations other than measurement on
a quantum computer must be reversible and copying an arbitrary register is impossible\ref{subsec:no-cloning}.
Neither of these restrictions appears for classical computers and as such while there are quantum algorithms where
quantum computers vastly outperform classical computers there are equally classical algorithms where quantum computers
are far less efficient than classical computers, though every classical algorithm has a quantum equivalent % TODO ref or citefor equivalence

\subsection{Practical differences between classical and quantum computers} \label{subsec:diffclassquantpracticey}
Not only are there inherent differences in the practical aspects between classical and quantum computers, there are also
differences induced by the fact that while classical computing has matured over the past decades, quantum computers are
still in the experimental phase.

One aspect where the two paradigms differ is scalability.
While it is possible to simply increase the number of gates of a classical computers, albeit perhaps sacrificing some
performance per gate and latency to heat dissipation concerns, to produce a more powerful classical computer a quantum
computer requires its system to remain in coherence\cite{find explaination}, which makes any scaling of the system a
complex tasks for physicists in addition to making simply connecting two quantum computers into one larger one impossible.

Another aspect is energy efficiency.
For a classical computer every single state needs a gate, whose operation comes with a small, but quickly compounding
cost in heat loss.
Indeed, heat dissipation concerns are the primary limiters for processor size.
While a quantum computer in theory also produces some waste heat per gate, its gates also operate on a superposition on
states and as such could theoretically be far more energy efficient.
In practice this advantage is however entirely counteracted by the fact that current quantum computers require extremely
low temperatures and as such spend most of their energy not on their actual operation, but on the cooling systems they
require to function.
Should a way to keep a quantum computer stable at room temperature be discovered however, quantum computers would easily
change from less to more energy efficient than classical computers.

The final problem that faces quantum computers that is largely solved by classical computers is error correction.
In a classical computer the occasional errors in gate operation, ironically due to quantum mechanical effects, is
corrected by duplicating the calculation and taking a majority vote of the result.
The no-cloning-theorem\ref{subsec:no-cloning} however prevents quantum computers from copying this approach instead
other algorithms have to be used to avoid the final practical superposition differing too much from the theoretical
superposition determined by the intended algorithm. % TODO ref to error correction chapter

