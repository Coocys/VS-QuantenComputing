\section{Einleitung}
\label{sec:einleitung}

Die Quanteninformatik ist ein schnell wachsendes Forschungsfeld. \footnote{Homeister, 2022, S. XII.}
Anders als klassische Computer, die auf Bits basieren und nur die Zustände 0 und 1 kennen, nutzt die Quanteninformatik sogenannte Quantenbits (Qubits), die in einer Superposition beide Zustände gleichzeitig annehmen können.
Dies führt zu einer völlig neuen Art der Datenverarbeitung, die in vielen Bereichen eine deutlich höhere Effizienz verspricht und potenziell über die Fähigkeiten heutiger Computersysteme hinausgeht.\\

In den letzten Jahren wurde vielversprechende Forschung zu verschiedenen Aspekten der Quanteninformatik durchgeführt, wie der Implementierung von Quantenalgorithmen, der Nutzung von Verschränkung und der Fehlertoleranz in Quantencomputern.
Diese Entwicklungen eröffnen neue Möglichkeiten unter anderem in Bereichen wie Kryptografie, Simulation chemischer Reaktionen und maschinelles Lernen.\footnote{Vgl. Fraunhofer Cluster of Excellence Cognitive Internet Technologies. \textit{Quantencomputing – Forschungsthemen}. cit.fraunhofer.de., \href{https://www.cit.fraunhofer.de/de/Forschungsthemen/quantencomputing.html} (abgerufen am 02. März 2025)}\\

In dieser Arbeit werden wir die Grundlagen der Quanteninformatik erläutern und einige der wichtigsten Konzepte und Prinzipien vorstellen.
Wir beginnen mit einer Einführung und erklären in welchen Bereichen sich ein Quantencomputer von einem klassischen Computer unterscheidet.
Dazu gehören Quantenbits und die Superposition, sowie Quantenregister und die Funktion der Verschränkung.
Außerdem werden wir die Bedeutung von Quantengattern und Quantenschaltkreisen erläutern und einen kurzen Umriss der uns jetzt schon bekannten Grenzen sowie zukünftiger Anwendungsgebiete geben.
Nach den Grundlagen kommen wir zu weiterführenden Themen, wie dem No Cloning Theorem und der Quantenteleportation.
Wie funktioniert die Quantenteleportation und was sind Herausforderungen?
Als letztes Thema werden wir uns mit der Quantenhardware beschäftigen, mit besonderem Fokus auf universellen Quantencomputern und der Error Correction.

