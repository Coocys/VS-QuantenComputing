\section{Schluss}
\label{sec:schluss}

Zusammenfassend lässt sich festhalten, dass die Quanteninformatik ein vielversprechendes Feld darstellt, dass das Potenzial hat, klassische Computer in zahlreichen Bereichen zu übertreffen.
So haben wir in den Grundlagen die fundamentalen Prinzipien und Funktionsweisen für Quantencomputer herausgestellt.
Mit Quantenbits ist es möglich, mehrere Zustände gleichzeitig zu repräsentieren was zusammen mit der Zerstörung der Superposition durch Messungen einen entscheidenden Vorteil in der Kryptografie bietet.
Die Verschränkung ermöglicht es, dass zwei Quantenbits in einem Zustand sind, sodass die Messung eines Bits den Zustand des anderen beeinflusst, was die Grundlage für Quantenteleportation bildet.\\

Mit dem No Cloning Theorem haben wir eine der grundlegendsten Einschränkungen und gleichzeitig auch Möglichkeit in der Quanteninformatik kennengelernt.
Es besagt, dass ein unbekannter Quantenzustand nicht kopiert werden kann.
Zum einen erfordert dies eine neue Denkweise bei der Entwicklung von Algorithmen, zum anderen bietet es die Möglichkeit, Quantenkommunikation abhörsicher zu gestalten.\\

Die Quantenteleportation ist eines der faszinierendsten Konzepte der Quanteninformatik, welches die Übertragung von Informationen ohne physische Bewegung ermöglicht.
Hier liegt die Herausforderung in der Erzeugung und Aufrechterhaltung der Quantenverschränkung, sowie der Notwendigkeit der klassischen Kommunikation.\\

Eine der größten technischen Hürden bildet die Dekohärenz, die durch die Wechselwirkung der Quantenbits mit ihrer Umgebung entsteht.
Es wurde beleuchtet, wie sie entsteht und berechnet wird, sowie wie sie durch Fehlerkorrekturverfahren minimiert werden kann.
Außerdem wurden unterschiedliche Modelle für universelle Quantencomputer vorgestellt, die auf verschiedenen Technologien basieren.\\

Zusammenfassend lässt sich sagen, dass die Quanteninformatik sowohl ein enormes Potenzial als auch einige fundamentale Herausforderungen birgt.
Die Entwicklung von Quantencomputern und -algorithmen ist ein aktives Forschungsfeld, das in den nächsten Jahren weiter an Bedeutung gewinnen wird.
Wir sind sehr gespannt, wie sich die Technologie entwickeln wird und in welchen Bereichen wir selbst damit auch in Berührung kommen werden.









