\section{Quantenbits} 
\label{sec:Quantenbits}
Ein Quantenbit, im Folgenden auch Qubit genannt, ist das Medium und die kleinste Einheit, auf dem in der Quanteninformatik gerechnet wird. Auf die genaue physische Realisierung wird in einem späteren Abschnitt der Quantenhardware eingegangen. Bis dahin reicht es das Quantenbit als eine Art Computer-Bit zu verstehen, das sich in einer sogenannten Superposition befindet. Im Zustand der Superposition kann es gleichzeitig den Wert ‚0‘ und ‚1‘ annehmen. Ein Quantenbit bleibt in dieser Superposition, bis es gemessen wird, woraufhin es diesen Zustand verliert und einen der Werte 0 und 1 annimmt. Die Wahrscheinlichkeit, mit der ein Quantenbit in den einen oder anderen Wert zerfällt, muss nicht gleichverteilt sein und kann beeinflusst werden. Dies macht Berechnungen auf Qbits erst möglich.
Mathematisch betrachtet, werden die Zustände ‚0‘ und ‚1‘ in der Quanteninformatik als Vektoren \verb|\(\binom{1}{0}\) \equiv| 0 und \verb|\(\binom{0}{1}\) \equiv| 1 dargestellt. Für die einfache Lesbarkeit werden diese Vektoren in der Quanteninformatik häufig in der sog. Ket-Darstellung beschrieben. 
Also \verb|\(\binom{1}{0}\)| \verb|\equiv| |0⟩ und \verb|\(\binom{0}{1}\)| \verb|\equiv| |1⟩. 
Um die Wahrscheinlichkeit zu beschreiben, welchen der beiden Werte ein Quantenbit nach der Messung annehmen wird, wird beiden Werten eine Amplitude \verb|\alpha| oder \verb|\beta| zugeordnet. Demnach wird ein Quantenbit in einer beliebigen Superposition als \verb|\alpha\cdot||0⟩+\verb|\beta\cdot||1⟩ dargestellt. \verb|\alpha| und \verb|\beta| sind komplexe Zahlen für die \verb|$\left|\alpha\right|^2+\left|\beta\right|^2=1$| gilt. 
Ein Qbit, das nach der Messung mit gleicher Wahrscheinlichkeit in einen der beiden Zustände 0 und 1 zerfällt, würde folglich als \frac{1}{\sqrt2}\bullet\left|0\right\rangle+\frac{1}{\sqrt2}\bullet\left|1\right\rangle dargestellt werden. Dabei gilt α = β = \frac{1}{\sqrt2} und erfüllt die Bedingung \left|\frac{1}{\sqrt2}\ \right|^2+\left|\frac{1}{\sqrt2}\ \right|^2=1 .
Ein Vektor \binom{\alpha}{\beta} wird als „Zustandsvektor“ bezeichnet und stellt den Zustand der Superposition eines Qubits dar. Die Bedingung \left|\alpha\right|^2+\left|\beta\right|^2=1 sorgt dafür, dass der Zustandsvektor immer ein Einheitsvektor ist. Dadurch kann jeder Zustand eines Qubits auf dem Einheitskreis eines zweidimensionalen Vektorsystems dargestellt werden. Somit kann ein Quantenbit unendlich viele Zustände haben, die auf einen definierten Bereich abgebildet werden können. 
Zum Zeitpunkt, als wir uns das Grundwissen erarbeitet haben, war uns nicht klar, weshalb α und β komplexe Zahlen sein sollten und warum sie „Amplituden“ und nicht „Wahrscheinlichkeiten“ oder Ähnliches genannt wurden. Wir gingen davon aus, dass wir früher oder später auf einen Use Case stoßen würden, in denen komplexe Zahlen und Amplituden wichtig werden würden. Dies war allerdings nur für die letzte Frage bedingt der Fall. Um kurz vorzugreifen: es gibt Algorithmen, in denen die Phase eines Zustandsvektors (also der Winkel des Zustandsvektors ausgehend vom Ursprung relativ zu einer der Achsen) eine wichtige Rolle spielt. Daher Amplituden? Kp.. Später kommt Sinus und Cosinus vor   

Wieso komplexe Zahlen? Weil Physik: siehe: https://www.cs.utep.edu/vladik/2023/tr23-44.pdf Man rechnet in Quanteninformatik eh nur mit reellen Zahlen.

Um mit Quantenbits rechnen zu können, muss man den Zustand eines Quantenbits verändern können. Wie dies physikalisch umgesetzt wird, wird später beschrieben. Aus mathematischer Sicht geschieht dies über unitäre 2x2 Matrizen. 
Eine Matrix A ist dann unitär, wenn ihre inverse Matrix A^{-1} gleich ihrer adjungierten Matrix A^\dag ist. Adjungiert ist eine Matrix A^\dag dann, wenn die Matrix A komplex konjugiert – also jedes Element der Matrix z_{ij}=\ a+ib zu z_{ij}^\ast\ =\ a\ – ib komplex konjugiert – und die Matrix dann transponiert wird. Für die Quanteninformatik reicht oft die Bedingung A^{-1}=A^T, da meistens mit α und β gerechnet wird, als seien sie reelle Zahlen. Durch diese Bedingung haben unitäre Matrizen die Eigenschaft, dass wenn sie zwei Mal mit einem Zustandsvektor verrechnet werden, der Zustandsvektor unverändert bleibt. Unitäre Transformationen sind also reversibel. Diese Bedingung ist notwendig, um die Länge der Zustandsvektoren beizubehalten.
Beispielhaft kann dies an einer Matrix B\ =\ \left(\begin{matrix}\frac{1}{2}&\frac{\sqrt3}{2}\\\frac{\sqrt3}{2}&-\frac{1}{2}\\\end{matrix}\right) demonstriert werden, die ein Qubit im Zustand|0⟩ in den Zustand \frac{1}{2}\left|\left.0\right\rangle\right.+\frac{\sqrt3}{2}\left|\left.1\right\rangle\right.  versetzt. Dies wird wie folgt berechnet:
B\left|\left.0\right\rangle\ =\ \right.\left(\begin{matrix}\frac{1}{2}&\frac{\sqrt3}{2}\\\frac{\sqrt3}{2}&-\frac{1}{2}\\\end{matrix}\right)\bullet\binom{1}{0}=\ \binom{\frac{1}{2}}{\frac{\sqrt3}{2}}
Daraus ergibt sich: \alpha\ =\ \frac{1}{2} und \beta\ =\ \frac{\sqrt3}{2}. Die Bedingung \left|\alpha\right|^2+\left|\beta\right|^2=1 trifft für beide Zustandsvektoren \binom{1}{0} und \binom{\frac{1}{2}}{\frac{\sqrt3}{2}} zu. Zudem ist B unitär, da B=\ B^T\ =\ B^{-1}, und daher eine zulässige Transformation.
Wenn man die Transformation B erneut an, ergibt sich folgendes:
B\binom{\frac{1}{2}}{\frac{\sqrt3}{2}}=\left(\begin{matrix}\frac{1}{2}&\frac{\sqrt3}{2}\\\frac{\sqrt3}{2}&-\frac{1}{2}\\\end{matrix}\right)\bullet\binom{\frac{1}{2}}{\frac{\sqrt3}{2}}=\ \binom{1}{0}.
Der Ursprungszustand ist mit der zweiten Anwendung der Transformation wieder hergestellt. Dies ist eine der Grundprinzipien der Quanteninformatik.
Hadamard-Matrix
Eine unitäre Transformation, die im Laufe unseres Lernprozesses immer wieder vorgekommen ist, ist die sog. Hadamard-Matrix H=\left(\begin{matrix}\frac{1}{\sqrt2}&\frac{1}{\sqrt2}\\\frac{1}{\sqrt2}&-\frac{1}{\sqrt2}\\\end{matrix}\right). Multipliziert mit dem Zustand |0⟩ transformiert sie ihn in den Zustand \frac{1}{\sqrt2}(\left|\left.0\right\rangle+\right.\left|\left.1\right\rangle)\right. und mit |1⟩ transformiert sie ihn in den Zustand \frac{1}{\sqrt2}(\left|\left.0\right\rangle-\right.\left|\left.1\right\rangle)\right.. Beide Folgezustände haben die gleiche Wahrscheinlichkeit bei der Messung in einen der Zustände |0⟩ oder |1⟩ zu zerfallen.
Damit ließe sich beispielsweise ein Münzwurf simulieren. Der Schaltkreis eines Münzwurfs sieht folgendermaßen aus: [hier Bild vom Münzwurfschaltkreis einfügen] Schaltkreise eigenen sich dafür Quantenalgorithmen anschaulich darzustellen. Auf sie wird später genauer eingegangen.
Zuerst wird ein Quantenbit in den Zustand |0⟩ gebracht. (Man könnte es auch in den Zustand |1⟩ bringen. Das ist für diesen Algorithmus unwichtig.) Danach wendet man die Hadamard-Transformation darauf an, um das Quantenbit in eine Superposition zu bringen, in der |0⟩ und |1⟩ mit gleicher Wahrscheinlichkeit vorkommen. Anschließend wird gemessen und die Superposition zerfällt zufällig in einen der Zustände |0⟩ oder |1⟩.
Formal beschrieben sieht der Algorithmus so aus:
	\left|\left.x\right\rangle\right.\ \gets\ \left|\left.0\right\rangle\right.\ 
	\left|\left.x\right\rangle\right.\ \gets\ H\left|\left.x\right\rangle\right.
	Miss\left|\left.x\right\rangle\right.
„|x⟩“ ist dabei die Bezeichnung des Quantenbits auf dem gerechnet wird.
Mit diesem Algorithmus ist möglich echte Zufallszahlen zu generieren, da es physikalisch unmöglich ist vorauszusagen, in welchen Zustand die Superposition zerfallen wird. Im Gegensatz zu einem echten Münzwurf, bei dem man das Wurfergebnis theoretisch berechnen könnte, wenn man sämtliche Variablen, wie z.B. Wurfhöhe, Drehmoment der Münze, Luftwiederstand, ect. kennen würde. Oder im Gegensatz zu einem herkömmlichen Computer, der nur Pseudozufallszahlen generieren kann. 
Quantenregister
Ein Quantenregister ist eine Aneinanderreihung mehrerer voneinander unabhängiger Quantenbits. Diese werden benötigt, um Quantenschaltkreise zu realisieren, damit auf ihnen logische Operationen durchgeführt werden können. 
Der Zustand eines Quantenregisters der Länge m wird als m-faches Tensorprodukt aller Zustände der einzelnen Quantenbits des Registers dargestellt. Ein Quantenregister R mit 2 Qubits \left|\left.x_0\right\rangle\ \right.und \left|\left.x_1\right\rangle\ \right.in beispielsweise den Basiszuständen \left|\left.x_0\right\rangle\ \right.=\binom{0}{1} und \left|\left.x_1\right\rangle\ \right.=\binom{1}{0} befindet sich im Zustand:
R\ =\ \left|\left.x_0\right\rangle\ \right.\otimes\left|\left.x_1\right\rangle\ \right.=\binom{0}{1}\otimes\binom{1}{0}=\left(\begin{matrix}0\\\begin{matrix}0\\\begin{matrix}1\\0\\\end{matrix}\\\end{matrix}\\\end{matrix}\right)
Alternativ würde man R =|1⟩ ⊗ |0⟩ = |10⟩ schreiben. Manchmal werden die Basiszustände zur Übersichtlichkeit auch in dezimalform dargestellt. |10⟩ wäre demnach |2⟩. 
Wie auch die einzelnen Qubits, kann sich das gesamte Quantenregister in einer Superposition befinden. Sind die Qubits aus dem obigen Beispiel in den Zuständen 
\left|\left.x_0\right\rangle\right.=\beta_0\left|\left.0\right\rangle\ \right.+\beta_1\left|\left.1\right\rangle\ \right.und \left|\left.x_1\right\rangle\right.=\gamma_0\left|\left.0\right\rangle\ \right.+\gamma_1\left|\left.1\right\rangle\ \right.
ist der Zustand des Quantenregisters 
R=\ \left|\left.x_0\right\rangle\right.\left|\left.x_1\right\rangle\right.=\left(\beta_0\left|\left.0\right\rangle\ \right.+\beta_1\left|\left.1\right\rangle\ \right.\right)\bullet(\gamma_0\left|\left.0\right\rangle\ \right.+\gamma_1\left|\left.1\right\rangle\ \right.)\ =\beta_0\gamma_0\left|\left.0\right\rangle\ \right.\left|\left.0\right\rangle\ \right.+\beta_0\gamma_1\left|\left.0\right\rangle\ \left|\left.1\right\rangle\ \right.\right.+\ \beta_1\gamma_0\left|\left.1\right\rangle\ \right.\left|\left.0\right\rangle\ \right.+\beta_1\gamma_1\left|\left.1\right\rangle\ \right.\left|\left.1\right\rangle\ \right.
Substituiert man \beta_i\gamma_j=\alpha_{ij} ergibt sich der Zustand
R=\alpha_{00}\left|\left.0\right\rangle\ \right.\left|\left.0\right\rangle\ \right.+\alpha_{01}\left|\left.0\right\rangle\ \left|\left.1\right\rangle\ \right.\right.+\ \alpha_{10}\left|\left.1\right\rangle\ \right.\left|\left.0\right\rangle\ \right.+\alpha_{11}\left|\left.1\right\rangle\ \right.\left|\left.1\right\rangle\ \right.
Und kann als
R=\alpha_{00}\left|\left.00\right\rangle\ \right.+\alpha_{01}\left|\left.01\right\rangle\right.+\ \alpha_{10}\left|\left.10\right\rangle\ \right.+\alpha_{11}\left|\left.11\right\rangle\ \right.
Oder
R=\alpha_0\left|\left.0\right\rangle\ \right.+\alpha_1\left|\left.1\right\rangle\right.+\ \alpha_2\left|\left.2\right\rangle\ \right.+\alpha_3\left|\left.3\right\rangle\ \right.
Geschrieben werden.
Da aus \left|\beta_0\right|^2+\left|\beta_1\right|^2=1 und \left|\gamma_0\right|^2+\left|\gamma_1\right|^2=1 sich \left|\alpha_{00}\right|^2+\left|\alpha_{01}\right|^2+\left|\alpha_{10}\right|^2+\left|\alpha_{11}\right|^2=1 ergibt, bildet α die Amplitude für den jeweiligen Zustand \left|\left.00\right\rangle\right.,\left|\left.01\right\rangle\right.,\left|\left.10\right\rangle\right.\ und\ \left|\left.11\right\rangle\right.. 
Allgemeiner gefasst befindet sich ein Quantenregister R der Länge n im Zustand 
R=\sum_{i=0}^{2^n-1}{\alpha_i\left|\left.i\right\rangle\right.}.
Für den die Bedingung
\sum_{i=0}^{2^n-1}\left|\alpha_i\right|^2=1
Gilt. Dabei entspricht i=0,...,2^n-1 der Dezimaldarstellung der Bits im Quantenregister und \left|\alpha_i\right|^2 der Wahrscheinlichkeit, dass sich das Register nach einer Messung im jeweiligen Zustand \left|\left.i\right\rangle\right. befindet.
Es ist möglich Transformationen nicht nur auf einzelnen Quantenbits durchzuführen, sondern auch auf ganze Register. Um eine Transformation auf einem Register eine Transformation durchzuführen, muss zuerst ein möglicherweise mehrfaches Tensorprodukt der Transformationsmatrix mit sich selbst berechnet werden. Da ein Tensorprodukt nur zwischen Matrizen derselben Größe berechnet werden kann, ist es nur möglich Transformationen auf 2^n langen Registern durchzuführen. 
Wir vermuten, dass wenn man eine Transformation auf ein Register der Länge m anwenden möchte, wobei m nicht in der Menge ist, die mit 2^n abgebildet werden kann, man das Register in zwei Unterregister aufteilen kann. Und die Transformation auf die Unterregister ausführt (Länge 2^max(n) < m und Länge Rest..) Da Rechenschritte physikalisch nur lokal durchgeführt werden können (s.31) scheint es für uns keinen Unterschied zu machen, auf wie vielen Qubits eine Transformation mathematisch durchgeführt wird.
Die Transformationen A_1,\ldots,A_{2^n} auf die Bits \left|\left.x_1\right\rangle\right.,\ldots,\left|\left.x_{2^n}\right\rangle\right. mit jeweils A_i\ auf \left|\left.x_i\right\rangle\right. entsprechen also der Transformation A_1\otimes,\ ...\ ,\otimes A_{2^n}\ auf das Register \left|\left.x_1,...,x_{2^n}\right\rangle\right..
Möchte man die Hadamard-Transformation H auf ein Register R = |00⟩ anwenden, müsste man zuerst die Hadamard-Transformation auf Registerebene 
H\otimes H\ =\ H_2=\ \frac{1}{\sqrt2}\left(\begin{matrix}H&H\\H&-H\\\end{matrix}\right)=\frac{1}{2}\left(\begin{matrix}\begin{matrix}1&1\\1&-1\\\end{matrix}&\begin{matrix}1&1\\1&-1\\\end{matrix}\\\begin{matrix}1&1\\1&-1\\\end{matrix}&\begin{matrix}-1&-1\\-1&1\\\end{matrix}\\\end{matrix}\right)
bilden. Daraus ergibt sich
H_2R\ =\frac{1}{2}\left(\begin{matrix}\begin{matrix}1&1\\1&-1\\\end{matrix}&\begin{matrix}1&1\\1&-1\\\end{matrix}\\\begin{matrix}1&1\\1&-1\\\end{matrix}&\begin{matrix}-1&-1\\-1&1\\\end{matrix}\\\end{matrix}\right)\binom{\begin{matrix}1\\0\\\end{matrix}}{\begin{matrix}0\\0\\\end{matrix}}=\binom{\begin{matrix}\frac{1}{2}\\\frac{1}{2}\\\end{matrix}}{\begin{matrix}\frac{1}{2}\\\frac{1}{2}\\\end{matrix}}\ 
Beziehungsweise \frac{1}{2}\left(\left|\left.00\right\rangle\right.+\left|\left.01\right\rangle\right.+\left|\left.10\right\rangle\right.+\left|\left.11\right\rangle\right.\right) oder \frac{1}{2}\left(\left|\left.0\right\rangle\right.+\left|\left.1\right\rangle\right.+\left|\left.2\right\rangle\right.+\left|\left.3\right\rangle\right.\right).
Dabei bleibt die Bedingung 
\sum_{i=0}^{2^n-1}\left|\alpha_i\right|^2=1
 Mit n = 2 und \alpha_i\ =\ \frac{1}{2}\ \forall\ i erfüllt. Die Wahrscheinlichkeit, dass jeder Basiszustand nach der Messung auftritt, ist gleichverteilt. Diese Berechnung kann genutzt werden, um echte zufallszahlen zwischen 0 und 3 zu generieren. Formal beschrieben sieht der Algorithmus aus, wie folgt:
	R=\left|\left.x_1x_0\right\rangle\right.\gets\left|\left.00\right\rangle\right.
	R=H_2R
	Miss R
Dieser Algorithmus kann auf eine beliebige Registergröße 2^n erweitert werden. 
Allerdings ist das Rechnen auf Registerebene bisher nur mathematisch sinnvoll. Tatsächlich werden sämtliche Rechenschritte in lokalen unitären Transformationen durchgeführt. „Lokal“ heißt in diesem Fall, dass maximal drei Qubits an der Berechnung beteiligt sind, da es physikalisch einfacher ist Transformationen auf drei Bits auszuführen als auf 2^n mit beliebig hohen n. Zudem sind mindestens drei Bit notwendig, um klassische Rechenverfahren in Quantenalgorithmen zu überführen. Dies wird später deutlich.
In den beiden Münzwurfbeispielen wurde die Hadamard-Transformation nur auf Quantenbits oder -register angewandt, bei denen sich alle Bits im Zustand |0⟩ befunden haben. Ein n langes Quantenregister R, dessen Bits sich alle im Zustand |0⟩ befunden haben und auf das die Hadamard-Transformation angewandt wurde, kann wie folgt dargestellt werden:
R=\frac{1}{\sqrt{2^n}}\sum_{i=0}^{2^n-1}\left|\left.i\right\rangle\right.
Diese Darstellung reicht allerdings nicht aus, um Quantenregister abzubilden, dessen Quantenbits sich vor der Hadamard-Transformation teilweise im Zustand |1⟩ befunden haben. Wendet man die Hadamard-Transformation auf ein Register |xy⟩ an, das sich im Zustand |01⟩ befindet, sähe das Quantenregister vor dem Ausmultiplizieren wie folgt aus:
\left|\left.01\right\rangle\right.{\buildrelH_2\frac\rightarrow}\frac{1}{\sqrt2}(\left|\left.0\right\rangle\right.+\left|\left.1\right\rangle\right.)\bullet\frac{1}{\sqrt2}(\left|\left.0\right\rangle\right.-\left|\left.1\right\rangle\right.).
Man kann an dieser Stelle die Information, ob sich die jeweiligen Quantenbits vorher im Zustand |0⟩ oder |1⟩ befunden haben, in das Vorzeichen ziehen:
\left|\left.xy\right\rangle\right.{\buildrelH_2\frac\rightarrow}\frac{1}{\sqrt2}(\left|\left.0\right\rangle\right.+{(-1)}^x\left|\left.1\right\rangle\right.)\bullet\frac{1}{\sqrt2}(\left|\left.0\right\rangle\right.+{(-1)}^y\left|\left.1\right\rangle\right.).
Ausmultipliziert ergibt dies: 
\frac{1}{2}\left(\left|\left.00\right\rangle\right.+\left(-1\right)^x\left|\left.01\right\rangle+\left(-1\right)^y\left|\left.10\right\rangle\right.+\left(-1\right)^{x\oplusy}\left|\left.11\right\rangle\right.\right.\right).\bigmAllgemeiner gefasst:
\frac{1}{2}\left(\left(-1\right)^{\left(0,0\right)\oplus z}\left|\left.00\right\rangle\right.+\left(-1\right)^{\left(0,1\right)\oplus z}\left|\left.01\right\rangle+\left(-1\right)^{\left(1,0\right)\oplus z}\left|\left.10\right\rangle\right.+\left(-1\right)^{\left(1,1\right)\oplus z}\left|\left.11\right\rangle\right.\right.\right),
mit z={(x,y)}^T.
Muss zugeben, den Sinn von diesem Ausmultiplizieren habe ich am Anfang nicht verstanden. Wahrscheinlich wird das bei komplexeren Algorithmen erst relevant.
Anhand der letzten Darstellung kann man ein Quantenregister der Lände n im Zustand
x\in{{0,1}}^n, auf das die Hadamard-Transformation angewandt wurde, wie folgt darstellen:
H_n\left|\left.x\right\rangle\right.=\frac{1}{\sqrt{2^n}}\sum_{y\in{0,1}^n}\left(-1\right)^{x\bullety}\left|\left.y\right\rangle\right..
Dabei ist x⋅y das Skalarprodukt \oplus_{i=1}^nx_iy_i der Vektoren x,y\in{{0,1}}^n. 
Auch die Hadamard-Transformation ist reversibel. Nehmen wir das vorherige Beispiel:
\left|\left.01\right\rangle\right.{\buildrelH_2\frac\rightarrow}\frac{1}{\sqrt2}(\left|\left.0\right\rangle\right.+\left|\left.1\right\rangle\right.)\bullet\frac{1}{\sqrt2}(\left|\left.0\right\rangle\right.-\left|\left.1\right\rangle\right.).
Wendet man die Hadamard-Transformation erneut an, ergibt sich:
\frac{1}{\sqrt2}\left(\left|\left.0\right\rangle\right.+\left|\left.1\right\rangle\right.\right)\bullet\frac{1}{\sqrt2}\left(\left|\left.0\right\rangle\right.-\left|\left.1\right\rangle\right.\right)\bigm{\buildrelH_2\frac\rightarrow}\frac{1}{\sqrt2}\left(\frac{1}{\sqrt2}\left(\left|\left.0\right\rangle\right.+\left|\left.1\right\rangle\right.\right)+\frac{1}{\sqrt2}\left(\left|\left.0\right\rangle\right.-\left|\left.1\right\rangle\right.\right)\right)\bullet\frac{1}{\sqrt2}\left(\frac{1}{\sqrt2}\left(\left|\left.0\right\rangle\right.+\left|\left.1\right\rangle\right.\right)-\frac{1}{\sqrt2}\left(\left|\left.0\right\rangle\right.-\left|\left.1\right\rangle\right.\right)\right)\bigm=\frac{1}{\sqrt2}\left(\frac{1}{\sqrt2}\left(\left|\left.0\right\rangle\right.+\left|\left.0\right\rangle\right.\right)\right)\bullet\frac{1}{\sqrt2}\left(\frac{1}{\sqrt2}\left(\left|\left.1\right\rangle\right.+\left|\left.1\right\rangle\right.\right)\right)\bigm=\frac{1}{2}\left(\left|\left.0\right\rangle\right.+\left|\left.0\right\rangle\right.\right)\bullet\frac{1}{2}\left(\left|\left.1\right\rangle\right.+\left|\left.1\right\rangle\right.\right)\ =\frac{1}{4}\left(\left|\left.01\right\rangle\right.+\left|\left.01\right\rangle\right.+\left|\left.01\right\rangle\right.+\left|\left.01\right\rangle\right.\right)=\left|\left.01\right\rangle\right..
Die Reversibilität der Hadamard-Transformation ist für komplexere Algorithmen von Bedeutung. Bevor auf diese und deren Darstellung in Form von Quantenschaltkreisen eingegangen werden kann, muss vorher noch auf den Vorgang der Messung eingegangen werden.
Messen
Die Messung „Miss|x⟩“ die einzige Transformation in der Quanteninformatik, die nicht reversibel / unitär ist. Beim Messen verliert das Quantenbit die Superposition und fällt zufällig, je nach Wahrscheinlichkeitsverteilung in einen der Zustände |0⟩ oder |1⟩. Aus diesen Zuständen ist nicht zu errechnen in welchem Zustand sich das Quantenbit vor der Messung befunden hat. Daher ist die Messung meist der letzte Schritt, den man für ein Quantenbit innerhalb eines Algorithmus durchführt. Das bedeutet allerdings nicht, dass die Messung aller Quantenbits immer am Ende eines Algorithmus geschehen muss. Es gibt Algorithmen in denen man zum Beispiel das Zwischenergebnis eines Quantenbits misst und anhand dieser Messung bestimmt, welche folgenden Transformationen für andere Quantenbits durchgeführt werden müssen. 

\end{document}