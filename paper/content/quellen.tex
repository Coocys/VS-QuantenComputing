\section{Quellenverzeichnis}
\label{sec:quellenverzeichnis}

Homeister, Matthias. \textit{Quanten Computing verstehen}. Springer Verlag, 2022, 6.Auflage.\\

Fraunhofer Cluster of Excellence Cognitive Internet Technologies. \textit{Quantencomputing – Forschungsthemen}. cit.fraunhofer.de., \href{https://www.cit.fraunhofer.de/de/Forschungsthemen/quantencomputing.html} (abgerufen am 02. März 2025)\\

Technische Universität Wien. \textit{Grenzen für Quantencomputer: Perfekte Uhren sind unmöglich}. tuwien.at., \href{https://www.tuwien.at/tu-wien/aktuelles/news/grenzen-fuer-quantencomputer-perfekte-uhren-sind-unmoeglich} (abgerufen am 02. März 2025)\\

Bundesamt für Sicherheit in der Informationstechnik. \textit{Quantenmechanische Sicherheitslücken (QML) - Studien}. bsi.bund.de., \href{https://www.bsi.bund.de/DE/Service-Navi/Publikationen/Studien/QML/QML_node.html} (abgerufen am 02. März 2025)\\

Fraunhofer-Institut für Werkstoffmechanik IWM. \textit{Quantencomputer für innovative Materialsimulation nutzen}. iwm.fraunhofer.de., \href{https://www.iwm.fraunhofer.de/de/geschaeftsfelder/werkstoffbewertung-lebensdauerkonzepte/materialmodellierung/quantencomputer_innovative_materialsimulation_nutzen.html} (abgerufen am 02. März 2025)\\