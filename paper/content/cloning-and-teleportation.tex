\section{Cloning und Teleportation} \label{sec:cloning-und-teleportation}
\subsection{No Cloning Theorem} \label{subsec:no-cloning}
One of the most intriguing aspects of quantum mechanics is the impossibility of
perfectly duplicating an arbitrary unknown quantum state.
This concept is encapsulated in the No-Cloning Theorem, a foundational result in quantum information theory.
The theorem has profound implications not only for quantum computing and quantum communication
but also for our understanding of the nature of information in quantum systems.

In classical physics, duplicating information is simple:
a copy machine can reproduce a document without altering the original.
However, in quantum mechanics, this straightforward act becomes a nontrivial and forbidden operation.
The No-Cloning Theorem asserts that there is no universal quantum operation
that can create an identical copy of an arbitrary unknown quantum state.

In this section, we will explore the No-Cloning Theorem from a variety of perspectives, discuss its proof,
examine its consequences, and understand how it shapes the landscape of quantum technologies.
\subsubsection{Definition}
The No-Cloning Theorem states that it is impossible to create an identical copy of an unknown quantum state.
Mathematically, there is no unitary operator UU that satisfies the following condition:
\begin{equation}
    U(\ket{\psi}\bigotimes\ket{e})=\ket{\psi}\bigotimes\ket{\psi}\label {eq:cloning_assertion}
\end{equation}
where $\ket{\psi}\ket{\psi}$ is an arbitrary quantum state and
$\ket{e}\ket{e}$ represents an auxiliary, blank state.
This equation expresses the idea that if we apply the operator UU to the state $\ket{\psi}\ket{\psi}$
(combined with an auxiliary state), the result should be two copies of the original state.
The theorem tells us that no such operator can exist for arbitrary $\ket{\psi}\ket{\psi}$,
because quantum information is fundamentally different from classical information.
\subsubsection{Proof}
The proof of the No-Cloning Theorem is typically framed in terms of unitary evolution
and the linear nature of quantum mechanics.
To explain this, consider a quantum state $\ket{\psi}\ket{\psi}$ and an auxiliary state $\ket{e}\ket{e}$.
Suppose we have a unitary operator UU that, when applied to
$\ket{\psi}\bigotimes\ket{e}\ket{\psi}\bigotimes\ket{e}$,
produces two copies of the state $\ket{\psi}\ket{\psi}$, as described above.

Now, imagine we apply the same operator UU to two different quantum states,
say $\ket{\psi_1}\ket{\psi_1}$ and $\ket{\psi_2}\ket{\psi_2}$, 
with the goal of producing two identical copies in each case:
\begin{equation}
    U(\ket{\psi_1}\bigotimes\ket{e})=\ket{\psi_1}\bigotimes\ket{\psi_1}
    U(\ket{\psi_2}\bigotimes\ket{e})=\ket{\psi_2}\bigotimes\ket{\psi_2} \label{eq:cloning_two_sates}
\end{equation}
But according to the linearity of quantum mechanics, the operator UU must satisfy:
\begin{equation}
    U(\alpha\ket{\psi_1}+\beta\ket{\psi_2})=\alpha U\ket{\psi_1}+\beta U\ket{\psi_2}
    \label{eq:cloning_linearity_distribution}
\end{equation}

% TODO make proof more clear
This leads to a contradiction because the action of UU on a superposition of states 
$\ket{\psi_1}\ket{\psi_1}$ and $\ket{\psi_2}\ket{\psi_2}$
cannot yield two identical copies of both states simultaneously.
The impossibility of creating an identical copy of arbitrary quantum states arises from the fact that
quantum states behave in a way that is fundamentally different from classical objects.

\subsubsection{Implications}
The No-Cloning Theorem has far-reaching implications for a number of fields in
quantum mechanics and quantum information science:

In quantum computing, the No-Cloning Theorem places fundamental limits on error correction.
While classical computers can easily replicate information to protect against errors
(e.g., by making redundant copies of data), this strategy is not possible in quantum computing.
Instead, quantum error correction schemes rely on entangling qubits in such a way that the information is spread across
multiple qubits in a manner that allows for recovery even if some of the qubits are disturbed or damaged.
However, these methods do not involve copying the quantum information itself.

One of the most practical consequences of the No-Cloning Theorem is in the realm of quantum cryptography,
specifically in quantum key distribution (QKD). In protocols like BB84, the theorem guarantees that an eavesdropper
cannot perfectly copy the quantum states transmitted between two parties.
This property ensures the security of quantum communication, as any attempt to intercept the quantum information
would necessarily disturb the system and alert the legitimate parties to the presence of an intruder.


\subsection{Verschränkte Teleportation}

- Braucht Setup (verschränktes Qubit teilen)

- überträgt einen Quantenzustand

- - Zustand am Ursprung wird zerstört (no cloning theorem)
