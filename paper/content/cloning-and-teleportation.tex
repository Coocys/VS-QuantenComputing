\section{No Cloning Theorem} \label{sec:no-cloning}
One of the most intriguing aspects of quantum mechanics is the impossibility of
perfectly duplicating an arbitrary unknown quantum state.
This concept is encapsulated in the No-Cloning Theorem, a foundational result in quantum information theory.
The theorem has profound implications not only for quantum computing and quantum communication
but also for our understanding of the nature of information in quantum systems.

In classical physics, duplicating information is simple:
a copy machine can reproduce a document without altering the original.
However, in quantum mechanics, this straightforward act becomes a nontrivial and forbidden operation.
The No-Cloning Theorem asserts that there is no universal quantum operation
that can create an identical copy of an arbitrary unknown quantum state.

In this section, we will explore the No-Cloning Theorem from a variety of perspectives, discuss its proof,
examine its consequences, and understand how it shapes the landscape of quantum technologies.
\subsection{Definition}\label{subsec:definition}
The No-Cloning Theorem states that it is impossible to create an identical copy of an unknown quantum state.
Mathematically, there is no unitary operator $U$ that satisfies the following condition:
\begin{equation}
    U(\ket{\psi}\bigotimes\ket{e})=\ket{\psi}\bigotimes\ket{\psi}\label {eq:cloning_assertion}
\end{equation}
where $\ket{\psi}$ is an arbitrary quantum state and
$\ket{e}$ represents an auxiliary, blank state.
This equation expresses the idea that if we apply the operator $U$ to the state $\ket{\psi}$
(combined with an auxiliary state), the result should be two copies of the original state.
The theorem tells us that no such operator can exist for arbitrary $\ket{\psi}$,
because quantum information is fundamentally different from classical information.
\subsection{Proof}\label{subsec:proof}
As a fundamental conclusion of quantum mechanics there have been several proofs of the theorem.
One such proof, a proof by contradiction chosen to be easy to follow, can be defined as follows.

First, we take two arbitrary quantum states, $\ket{\psi_1}$ and $\ket{\psi_2}$.
Next, both these states are cloned with the cloning operator $U$, which we defined in the previous section and now assume
exists for our proof by contradiction, and an auxiliary state $e$.

\begin{equation}
\begin{split}
    U(\ket{\psi_1} \bigotimes \ket{e}) = \ket{\psi_1} \bigotimes \ket{\psi_1}\\
    U(\ket{\psi_2} \bigotimes \ket{e}) = \ket{\psi_2} \bigotimes \ket{\psi_2}
\end{split}\label{eq:clone_two_states}
\end{equation}

Using this information, we now have two ways to write the scalar product of
$\braket{U(\psi_1 \bigotimes e)}{U(\psi_2 \bigotimes e)}$.
One uses the result of the previous equation
while the other uses the fact that quantum operations preserve the scalar product of their inputs\cite{findsource}.

\begin{equation}
\begin{split}
    \braket{U(\psi_1 \bigotimes e)}{U(\psi_2 \bigotimes e)} = \braket{\psi_1 \bigotimes \psi_1}{\psi_2 \bigotimes \psi_2}\\
    \braket{U(\psi_1 \bigotimes e)}{U(\psi_2 \bigotimes e)} = \braket{\psi_1 \bigotimes e}{\psi_2 \bigotimes e}
\end{split}\label{eq:two_scalar_products}
\end{equation}

Simple substitution leads then to the following equation:
\begin{equation}
    \braket{\psi_1 \bigotimes \psi_1}{\psi_2 \bigotimes \psi_2} =
    \braket{\psi_1 \bigotimes e}{\psi_2 \bigotimes e}
\label{eq:tensor_equals}
\end{equation}

Because the scalar and tensor products are compatible that further simplifies to:

\begin{equation}
    \braket{\psi_1}{\psi_2}\braket{\psi_1}{\psi_2}=\braket{\psi_1}{\psi_2}\braket{e}{e}
    \label{eq:scalar_equals}
\end{equation}

And finally, because for any state $\ket{e}$ the equation $\braket{e}{e} = 1$ holds we arrive at this equation:
\begin{equation}
    \braket{\psi_1}{\psi_2}^2 = \braket{\psi_1}{\psi_2}
    \label{eq:square_equals_single}
\end{equation}

It should now be obvious that this equation only has two solutions:
$\braket{\psi_1}{\psi_2} = 1$ or $\braket{\psi_1}{\psi_2} = 0$.
The first solution implies that $\psi_1 = \psi_2$, which does not help a general copying operation but does imply
an operation producing copies of a specific quantum state is possible.
While the second solution allows for $\psi_1$ and $\psi_2$ to differ and is as such more promising for a general operation
it also implies that the two states are orthogonal to each other which, while less restrictive than demanding that the
two states are equal is still to restrictive for a general copying operation while allowing some specialised copy operations.
\subsection{Implications}\label{subsec:implications}
The No-Cloning Theorem has far-reaching implications for a number of fields in
quantum mechanics and quantum information science:

In quantum computing, the No-Cloning Theorem places fundamental limits on error correction.
While classical computers can easily replicate information to protect against errors
(e.g., by making redundant copies of data), this strategy is not possible in quantum computing.
Instead, quantum error correction schemes rely on entangling qubits in such a way that the information is spread across
multiple qubits in a manner that allows for recovery even if some of the qubits are disturbed or damaged.
However, these methods do not involve copying the quantum information itself.

One of the most practical consequences of the No-Cloning Theorem is in the realm of quantum cryptography,
specifically in quantum key distribution (QKD). In protocols like BB84, the theorem guarantees that an eavesdropper
cannot perfectly copy the quantum states transmitted between two parties.
This property ensures the security of quantum communication, as any attempt to intercept the quantum information
would necessarily disturb the system and alert the legitimate parties to the presence of an intruder.


\section{Quantum Teleportation}\label{sec:quantum-teleportation}

\subsection{Introduction}\label{subsec:introduction}
Quantum teleportation is a groundbreaking phenomenon that allows the transfer of quantum information
between two distant locations without physically moving the particle or object carrying the information.
Unlike the theoretical classical teleportation, which involves the transportation of matter or energy,
quantum teleportation focuses on the transmission of quantum states.
This process leverages the principles of quantum mechanics, including entanglement, superposition,
and the no-cloning theorem which we explained in the previous section\ref{sec:no-cloning},
to transfer the state of a quantum object (such as a photon or an electron) from one place to another,
without regard for distance.

The term ``quantum teleportation'' can be somewhat misleading because, in the process, no actual matter is teleported.
Instead, what is ``teleported'' is the information about the quantum state of a particle.
The key to quantum teleportation is quantum entanglement,
a phenomenon where two or more particles become correlated in such a way that the state of one particle instantaneously
influences the state of the other, no matter how far apart they are.
This ``spooky action at a distance'', as Albert Einstein famously called it,
allows for the transfer of quantum information between distant parties without regard for speed-of-light delays.

\subsection{Setup}

\subsection{Measurement}

\subsection{State reconstruction}